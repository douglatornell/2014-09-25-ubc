\documentclass[xcolor=dvipsnames]{beamer}

\usetheme{Susan}
\usepackage{graphics}
\usepackage{multicol}
\usepackage{url}

\section*{Python for EOAS}

\begin{document}

\begin{frame}
\begin{center}{\Huge Python for EOAS}
\end{center}
Examples of using modules
\end{frame}

\begin{frame}
\frametitle{Learning Goals}
\begin{enumerate}
\item  Read CSV data from files into NumPy data structures,
using pandas
\item Use list boolean slices to select data
\item Use requests to get data from the web
\end{enumerate}
\end{frame}

\begin{frame}
\frametitle{Get the Data}
\begin{itemize}
\item Use your browser to go to \url{http://climate.weather.gc.ca/} and work your way through to the "Hourly Data Report" for yesterday at the *Vancouver Intl A* station.
\item Download the August 2013 hourly data as a CSV file
\item Use your shell skills to confirm that:
\begin{itemize}
    \item You really got a CSV file
    \item It's for the *Vancouver Intl A* station
    \item It contains hourly data for the whole month of August 2013
\end{itemize}
\item Move or copy the CSV file into the *data-explore/* directory in your repo and commit it.
\end{itemize}
\end{frame}

\begin{frame}
\begin{enumerate}
\item  Read CSV data from files into NumPy data structures,
using pandas
\end{enumerate}
\begin{itemize}
\item import numpy as np
\item import pandas as pd
\item !head eng-hourly-08012013-08312013.csv
\item !head -20 eng-hourly-08012013-08312013.csv
\item !tail eng-hourly-08012013-08312013.csv
\item data = pd.read\_csv('eng-hourly-08012013-08312013.csv', skiprows=16)
\item print data[0:4]
\item print data.tail(1)
\item !tail -1 eng-hourly-08012013-08312013.csv
\item data = pd.read\_csv('eng-hourly-08012013-08312013.csv', skiprows=16, encoding="ISO-8859-1")
\item print data.columns
\end{itemize}
\end{frame}

\begin{frame}[fragile]
\begin{enumerate}
\setcounter{enumi}{1}
\item Use list boolean slices to select data
\end{enumerate}
\begin{itemize}
\item temps = data[u'Temp (°C)']
\item print 'max:', temps.max(), 'on', data['Date/Time'][temps.argmax()]
\item print 'min:', temps.min(), 'on', data['Date/Time'][temps.argmin()]
\item print 'mean:', temps.mean()
\item print 'std dev:', temps.std()
\item {\small
\begin{verbatim}
for day in range(1, 32):
    mask = data['Day']==day
    max_temp = temps[mask].max()
    date = data[mask]['Date/Time'][temps[mask].argmax()][:11]
    hour = data[mask]['Time'][temps[mask].argmax()]    
    print 'max temperature on',date, 'was', max_temp, 'at', hr
\end{verbatim}}
\end{itemize}
\end{frame}

\begin{frame}
\frametitle{Exercise}
Plot the daily maximum temperature.
\end{frame}

\begin{frame}[fragile]
\begin{enumerate}
\setcounter{enumi}{2}
\item Use requests to get data from the web
\end{enumerate}
\begin{itemize}
\item import requests
\item url = 'http://climate.weather.gc.ca/climateData/bulkdata\_e.html'
\item 
\begin{verbatim}
params = {
    'timeframe': 1,
    'stationID': 51442,
    'Year': 2013,
    'Month': 7,
    'Day': 1,
    'format': 'csv',
}
\end{verbatim}
\item response = requests.get(url, params=params)
\item response.headers
\item from StringIO import StringIO
\item fakefile = StringIO(response.content)
\item datajul = pd.read\_csv(fakefile, skiprows=16, encoding="ISO-8859-1")
\item print datajul.head(2)
\end{itemize}
\end{frame}

\begin{frame}
\frametitle{Exercise}
Plot the daily maximum temperature for both August and July
\end{frame}

\end{document}