\documentclass[xcolor=dvipsnames]{beamer}
% Class options include: notes, notesonly, handout, trans,
%                        hidesubsections, shadesubsections,
%                        inrow, blue, red, grey, brown

% Theme for beamer presentation.
\usetheme{Susan}
\usepackage{graphics}
\usepackage{multicol}
\usepackage{url}

% Other themes include: beamerthemebars, beamerthemelined, 
%                       beamerthemetree, beamerthemetreebars  


\section*{Programs with Python)}

\begin{document}

\begin{frame}
\begin{center}{\Huge Building Programs with Python}
\end{center}
\end{frame}

\begin{frame} 
\frametitle{Analyzing Patient Data Part 1}
\begin{enumerate}
\item    Explain what a library is, and what libraries are used for.
\item    Load a Python library and use the things it contains.
\item    Read tabular data from a file into a program.
\item    Assign values to variables.
\item    Select individual values and subsections from data.
\end{enumerate}

\begin{multicols}{2}
\begin{itemize}
\item import numpy
\item numpy.loadtxt(fname=  delimiter=)
\item weight\_kg = 55
\item print
\item weight\_lb = 2.2 * weight\_kg
\item type(data)
\item data.shape
\item data[0,0], data[0:1,0:1]
\item data[0:10:2,1]
\item data[:3,36:]
\end{itemize}
\end{multicols}
\end{frame}

\begin{frame}
\frametitle{Analyzing Patient Data Part 2}
\begin{enumerate}
\setcounter{enumi}{5}
\item    Perform operations on arrays of data.    
\item    Display simple graphs.  
\end{enumerate}
\begin{multicols}{2}
\begin{itemize}
\item data.mean()
\item data.std()
\item data.mean(axis=0)
\item \%matplotlib inline
\item from matplotlib import pyplot 
\item pyplot.imshow(data)
\item pyplot.show()
\item pyplot.plot(ave\_inflammation)
\item import matplotlib import pyplot as plt
\item plt.subplot(1,3,1)
\item plt.ylabel('average')
\item plt.show()
\end{itemize}
\end{multicols}
\end{frame}


\begin{frame}
\frametitle{Exercise}
Create a single plot showing 1) the mean for each day and 2) the mean + 1 standard deviation for each day and the 3) the mean - 1 standard deviation for each day.
\end{frame}

\begin{frame}[fragile]
\frametitle{Creating Functions Part 1}
\begin{enumerate}
\item    Define a function that takes parameters.
\item    Return a value from a function.
\item    Test and debug a function.
\end{enumerate}
\begin{itemize}
\item 
\begin{verbatim} 
def fahr_to_kelvin(temp):
     return ((temp - 32) * (5/9)) + 273.15
\end{verbatim}
\item from \_\_future\_\_ import division
\item 
\begin{verbatim}
def kelvin_to_celsius(temp):
    return temp - 273.15
\end{verbatim}
\item 
\begin{verbatim}
def fahr_to_celsius(temp):
    temp_k = fahr_to_kelvin(temp)
    result = kelvin_to_celsius(temp_k)
    return result
\end{verbatim}
\end{itemize}
\end{frame}

\begin{frame}
\frametitle{Creating Functions Part 2}
\begin{enumerate}
\setcounter{enumi}{3}
%\item    Explain what a call stack is, and trace changes to the call stack as functions are called.
\item    Explain the scope of a variable and the idea of encapsulation.
\end{enumerate}
\resizebox{\textwidth}{!}{\includegraphics{img/python-call-stack-05.png}}
\end{frame}


\begin{frame}[fragile]
\frametitle{Creating Functions Part 3}
\begin{enumerate}
\setcounter{enumi}{2}
\item    Test and debug a function.
\item    Explain why we should divide programs into small, single-purpose functions.
\end{enumerate}
\begin{itemize}
\item 
\begin{verbatim} 
def centre(data, desired):
    return (data - data.mean()) + desired
\end{verbatim}
\item z = numpy.zeros((2,2))
\item print centre(z, 3)
\item print data.std() - centred.std()
\item '''centre(data, desired): return a new array containing the original data centered around the desired value.'''
\item help(centre)
\end{itemize}
\end{frame}

\begin{frame}
\frametitle{Exercise}
Write a function called analyze that takes a filename as a parameter and displays the three graphs produced in the previous lesson, i.e., analyze('inflammation-01.csv') should produce the graphs already shown, while analyze('inflammation-02.csv') should produce corresponding graphs for the second data set. Be sure to give your function a docstring. Hint: a function can just ``do'' something.  It doesn't necessarily need to return anything.
\end{frame}



\begin{frame}[fragile]
\frametitle{Creating Functions Part 4}
\begin{enumerate}
\setcounter{enumi}{5}
\item    Set default values for function parameters.
\end{enumerate}
\begin{itemize}
\item def center(data, desired = 0):
\item
\begin{verbatim}
def display(a=1, b=2, c=3):
    print 'a:', a, 'b:', b, 'c:', c

print 'no parameters:'
display()
print 'one parameter:'
display(55)
print 'two parameters:'
display(55, 66)
\end{verbatim}
\item help(numpy.loadtxt)
\end{itemize}
\end{frame}


\end{document}