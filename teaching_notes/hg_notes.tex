\documentclass[xcolor=dvipsnames]{beamer}
% Class options include: notes, notesonly, handout, trans,
%                        hidesubsections, shadesubsections,
%                        inrow, blue, red, grey, brown

% Theme for beamer presentation.
\usetheme{Susan}
\usepackage{graphics}
\usepackage{multicol}
\usepackage{url}

\section*{Mercurial (hg)}

\begin{document}

\begin{frame}
\begin{center}{\Huge Mercurial (hg)}
\end{center}
\end{frame}

\begin{frame}
\frametitle{Introduction}
Learning Goal
\begin{enumerate}

 \item   Explain when and why you should use version control

\end{enumerate}
\end{frame}

\begin{frame}
\begin{columns}
\column{0.6\textwidth}
\resizebox{!}{\textheight}{\includegraphics{img/phd101212s.png}}
\column{0.4\textwidth}
"Piled Higher and Deeper" by Jorge Cham, http://www.phdcomics.com
\end{columns}
\end{frame}

\begin{frame}[label=part1]
\frametitle{A Better Kind of Backup - Part 1}
%Learning Goals
\begin{enumerate}
\item Explain which initialization and configuration steps are required once per machine, and which are required once per repository.
\item Add files to Mercurial's collection of tracked files.
\item Go through the modify-commit cycle for single and multiple files and explain where information is stored before and after the commit.
\item Identify and use Mercurial revision numbers and changeset identifiers.
\item Compare files with previous version of themselves.
\end{enumerate}

\begin{multicols}{2}
\begin{itemize}
\item Mercurial.ini (windows)
\item ~/.hgrc (Linux/Mac)
\item mkdir planets
\item cd planets
\item hg init
\item ls -a
\item hg verify
\item nano mars.txt
\item hg status
\item hg add mars.txt
\item hg commit -m ``Starting...''
\item hg log
\item hg diff
\end{itemize}
\end{multicols}
\end{frame}

\begin{frame}[fragile]
\frametitle{Mercurial.ini for Windows}
Create a new file called \%USERPROFILE\%\textbackslash Mercurial.ini (that's spelled \$USERPROFILE/Mercurial.ini if you are in gitbash)
\begin{verbatim}
[ui]
username = Vlad Dracula <vlad@tran.sylvan.ia>
editor = nano

[extensions]
color =

[color]
mode = win32
\end{verbatim}
\end{frame}

\againframe{part1}

\begin{frame}[label=part2]
\frametitle{A Better Kind of Backup - Part 2}
\begin{enumerate}
\setcounter{enumi}{4}
\item Compare files with old versions of themselves.
\item Restore old versions of files.
\item Configure Mercurial to ignore specific files, and explain why it is sometimes useful to do so.
\end{enumerate}
\begin{multicols}{2}
\begin{itemize}
\item hg diff --rev 1:2 mars.txt
\item hg diff -r 0:2 mars.txt
\item hg diff --change 1
\item hg revert mars.txt
\item hg revert --rev 0 mars.txt
\item hg status
\item mkdir results
\item touch a.dat b.dat c.dat results/a.out results/b.out
\item hg status
\item nano .hgignore
\item hg status --ignored
\end{itemize}
\end{multicols}
\end{frame}

\begin{frame}[fragile]
\frametitle{.hgignore}
\begin{verbatim}
syntax: glob
*.dat
results/
\end{verbatim}
\end{frame}

\againframe{part2}

\begin{frame}
\frametitle{Exercise}
Create a new Mercurial repository on your computer called bio. Write a three-line biography for yourself in a file called me.txt, commit your changes, then modify one line and add a fourth and display the differences between its updated state and its original state.
\end{frame}


\begin{frame}
\frametitle{Collaborating}
\begin{enumerate}
  \item Explain what remote repositories are and why they are useful.
  \item Explain what happens when a remote repository is cloned.
  \item Explain what happens when changes are pushed to or pulled from a remote repository.
\end{enumerate}
\begin{multicols}{2}
\begin{itemize}
  \item hg paths
  \item hg push
  \item hg pull
  \item hg clone
  \item hg log --graph
  \item hg update
\end{itemize}
\end{multicols}
\end{frame}

\end{document}

