\documentclass[xcolor=dvipsnames]{beamer}
% Class options include: notes, notesonly, handout, trans,
%                        hidesubsections, shadesubsections,
%                        inrow, blue, red, grey, brown

% Theme for beamer presentation.
\usetheme{Susan}
\usepackage{graphics}
\usepackage{multicol}
\usepackage{url}

% Other themes include: beamerthemebars, beamerthemelined,
%                       beamerthemetree, beamerthemetreebars


\section*{Shell}

\begin{document}

\begin{frame}
\begin{center}{\Huge Shell}
\end{center}
\end{frame}

\begin{frame}
\frametitle{Introduction}
Learning Goals
\begin{enumerate}

 \item   Explain how the shell relates to the keyboard, the screen, the operating system, and users' programs.
 \item   Explain when and why command-line interfaces should be used instead of graphical interfaces.

\end{enumerate}

\end{frame}

\begin{frame}
\frametitle{Files and Directories}
%Learning Goals
\begin{enumerate}
\item    Explain the similarities and differences between a file and a directory.
\item    Translate an absolute path into a relative path and vice versa.
\item    Construct absolute and relative paths that identify specific files and directories.
\item    Explain the steps in the shell's read-run-print cycle.
\item    Identify actual command, flags, and filenames in command-line call.
\item    Demonstrate the use of tab completion, and explain its advantages.
\end{enumerate}

\begin{multicols}{2}
\begin{itemize}
\item whoami
\item pwd
\item /
\item ls
\item ls -F
\item ls -F data
\item ls -F /data
\item cd data
\item cd ..
\item ls -F -a
\item ls north-pacific-gyre/2012-07-03
\item ls no tab
\end{itemize}
\end{multicols}
\end{frame}


\begin{frame}[label=CreatingThings]
\frametitle{Creating Things}
\begin{enumerate}
\item    Create a directory hierarchy that matches a given diagram.
\item    Create files in that hierarchy using an editor or by copying and renaming existing files.
\item    Display the contents of a directory using the command line.
\item    Delete specified files and/or directories.
\end{enumerate}
\begin{multicols}{2}
\begin{itemize}
\item mkdir thesis
\item cd thesis
\item nano draft.txt
\item rm draft.txt
\item rm thesis
\item rmdir thesis
\item rm -r thesis
\item mv thesis/draft.txt thesis/quotes.txt
\item mv thesis/quotes.txt .
\item cp quotes.txt thesis/quotations.txt
\end{itemize}
\end{multicols}
\end{frame}


\begin{frame}[fragile]
Create a workspace on your desktop so that it's easy to find,
and easy to explore with your GUI filesystem tool
(Explorer, Finder, Nautilus, ...)
\begin{verbatim}
$ cd
$ cd Desktop
$ mkdir swc
$ cd swc
\end{verbatim}
\end{frame}


\begin{frame}
A bug in recent versions of nano on Windows causes the Git Bash terminal windows to be blanked when nano exits -- annoying.\\[20pt]

A work-around for the issue is to open another Git Bash window and run nano there.
Of course you will have to {\tt cd} in both windows as you move around the file system.\\[20pt]

An alternative is to download and install the Notepad++ editor and ask one of the helpers or instructors to help you add Notepad++ to your PATH -- the list of directories that the shell looks in to find the programs you ask it to run.
\end{frame}


\againframe{CreatingThings}


\begin{frame}[fragile]
\frametitle{Exercise}
What command(s) could you run so that the commands below will produce the output shown? (and do it)
\begin{verbatim}
$ ls
analyzed   raw
$ ls analyzed
fructose.dat    glucose.dat    sucrose.dat
\end{verbatim}
\end{frame}


\begin{frame}[label=PipesAndFilters]
\frametitle{Pipes and Filters}
\begin{enumerate}

\item    Redirect a command's output to a file.
\item    Process a file instead of keyboard input using redirection.
\item    Construct command pipelines with two or more stages.
\item    Explain what usually happens if a program or pipeline isn't given any input to process.
\item    Explain Unix's "small pieces, loosely joined" philosophy.

\end{enumerate}
\begin{multicols}{2}
\begin{itemize}
\item cd molecules
\item wc *.pdb
\item *, ?
\item wc -l
\item wc --help
\item wc -l *.pdb $>$ lengths
\item cat lengths
\item sort lengths
\item sort lengths $>$ sorted-lengths
\item head -1 sorted-lengths
\item sort lengths $|$ head -1
\item cd north-pacific-gyre/2012-07-03
\item wc -l *.txt
\item wc -l *.txt $|$ sort $|$ head -5
\item ls *Z.txt
\end{itemize}
\end{multicols}
\end{frame}


\begin{frame}[fragile]
We're going to start working with Nelle Nemo's Great Pacific Garbage Patch files,
so everybody needs a copy of her directories and files so that you can pretend that you are Nelle.

Use Mercurial to grab the files from Bitbucket and put them in a {\tt nnemo} directory in your SWC workspace:
{\footnotesize
\begin{verbatim}
$ cd
$ cd Desktop/swc
$ hg clone https://bitbucket.org/douglatornell/swc-nelle-files nnemo
\end{verbatim}
}

You can copy and paste the {\tt hg clone} command from the Etherpad.
We'll learn what it means in the Version Control with Mercurial section later today.
\end{frame}


\againframe{PipesAndFilters}


\begin{frame}
\frametitle{Loops - Part 1}
\begin{enumerate}
  \item Write a loop that applies one or more commands separately to each file in a set of files.
  \item Trace the values taken on by a loop variable during execution of the loop.
  \item Explain the difference between a variable's name and its value.
  \item Explain why spaces and some punctuation characters shouldn't be used in files' names.
\end{enumerate}
\begin{multicols}{2}
\begin{itemize}
  \item for ... do ... done
  \item varname, \$varname
  \item echo
  \item "\$varname"
\end{itemize}
\end{multicols}
\end{frame}


\begin{frame}
\frametitle{Loops - Part 2}
\begin{enumerate}
  \item Demonstrate how to see what commands have recently been executed.
  \item Re-run recently executed commands without retyping them.
\end{enumerate}
\begin{multicols}{2}
\begin{itemize}
  \item ls *[AB].txt
  \item Up-Arrow
  \item history
  \item Ctrl-A, Ctrl-E
  \item Ctrl-R
  \item Ctrl-C
\end{itemize}
\end{multicols}
\end{frame}


\begin{frame}[fragile]
\frametitle{Exercise}
In your {\tt analyzed} directory, what is the effect of this loop?

\begin{verbatim}
for sugar in *.dat
do
    echo $sugar
    cat $sugar > xylose.dat
done
\end{verbatim}
\begin{enumerate}
  \item Prints {\tt fructose.dat}, {\tt glucose}, and {\tt sucrose}, and copies {\tt sucrose} to create {\tt xylose}.
  \item Prints {\tt fructose}, {\tt glucose}, and {\tt sucrose}, and concatenates all three files to create {\tt xylose}.
  \item Prints {\tt fructose}, {\tt glucose}, {\tt sucrose}, and {\tt xylose}, and copies {\tt sucrose} to create {\tt xylose}.
  \item None of the above.
\end{enumerate}
\end{frame}


\begin{frame}
\frametitle{Shell Scripts}
\begin{enumerate}
\item Write a shell script that runs a command or series of commands for a fixed set of files.
\item Run a shell script from the command line.
\item Write a shell script that operates on a set of files defined by the user on the command line.
\item Create pipelines that include user-written shell scripts.
\end{enumerate}
\begin{multicols}{2}
\begin{itemize}
  \item bash myscript.sh
  \item \$1, \$2, ... \$n, \$*
  \item \# comment
  \item history $|$ tail -4 $>$ script.sh
\end{itemize}
\end{multicols}
\end{frame}


\begin{frame}[fragile]
\frametitle{Exercise}
Write a shell script called {\tt longest.sh} that takes the name of a directory and a filename extension as its parameters,
and prints out the number of lines,
directory,
and name of the file with the most lines in that directory with that extension.
For example:

\begin{verbatim}
$ bash longest.sh more-molecules pdb
\end{verbatim}

would print the number of lines,
directory,
and name of the {\tt .pdb} file in {\tt more-molecules} that has
the most lines.
\end{frame}


\begin{frame}
\frametitle{Finding Things}
\begin{enumerate}
  \item Use {\tt grep} to select lines from text files that match simple patterns.
  \item Use {\tt find} to find files whose names match simple patterns.
  \item Use the output of one command as the command-line parameters to another command.
  \item Explain what is meant by "text" and "binary" files, and why many common tools don't handle the latter well.
\end{enumerate}
\end{frame}

\end{document}
