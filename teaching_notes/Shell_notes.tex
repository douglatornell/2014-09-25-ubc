\documentclass[xcolor=dvipsnames]{beamer}
% Class options include: notes, notesonly, handout, trans,
%                        hidesubsections, shadesubsections,
%                        inrow, blue, red, grey, brown

% Theme for beamer presentation.
\usetheme{Susan}
\usepackage{graphics}
\usepackage{multicol}
\usepackage{url}

% Other themes include: beamerthemebars, beamerthemelined, 
%                       beamerthemetree, beamerthemetreebars  


\section*{Shell}

\begin{document}

\begin{frame}
\begin{center}{\Huge Shell}
\end{center}
\end{frame}

\begin{frame}
\frametitle{Introduction}
Learning Goals
\begin{enumerate}

 \item   Explain how the shell relates to the keyboard, the screen, the operating system, and users' programs.
 \item   Explain when and why command-line interfaces should be used instead of graphical interfaces.

\end{enumerate}

\end{frame}

\begin{frame}
\frametitle{Files and Directories}
%Learning Goals
\begin{enumerate}
\item    Explain the similarities and differences between a file and a directory.
\item    Translate an absolute path into a relative path and vice versa.
\item    Construct absolute and relative paths that identify specific files and directories.
\item    Explain the steps in the shell's read-run-print cycle.
\item    Identify actual command, flags, and filenames in command-line call.
\item    Demonstrate the use of tab completion, and explain its advantages.
\end{enumerate}

\begin{multicols}{2}
\begin{itemize}
\item whoami
\item pwd
\item /
\item ls
\item ls -F
\item ls -F data
\item ls -F /data
\item cd data
\item cd ..
\item ls -F -a
\item ls north-pacific-gyre/2012-07-03
\item ls no tab
\end{itemize}
\end{multicols}
\end{frame}

\begin{frame}
\frametitle{Creating Things}
\begin{enumerate}
\item    Create a directory hierarchy that matches a given diagram.
\item    Create files in that hierarchy using an editor or by copying and renaming existing files.
\item    Display the contents of a directory using the command line.
\item    Delete specified files and/or directories.
\end{enumerate}
\begin{multicols}{2}
\begin{itemize}
\item mkdir thesis
\item cd thesis
\item nano draft.txt
\item rm draft.txt
\item rm thesis
\item rmdir thesis
\item rm -r thesis
\item mv thesis/draft.txt thesis/quotes.txt
\item mv thesis/quotes.txt .
\item cp quotes.txt thesis/quotations.txt
\end{itemize}
\end{multicols}
\end{frame}

\begin{frame}[fragile]
\frametitle{Exercise}
What command(s) could you run so that the commands below will produce the output shown? (and do it)
\begin{verbatim}
$ ls
analyzed   raw
$ ls analyzed
fructose.dat    sucrose.dat
\end{verbatim}
\end{frame}

\begin{frame}
\frametitle{Pipes and Filters}
\begin{enumerate}

\item    Redirect a command's output to a file.
\item    Process a file instead of keyboard input using redirection.
\item    Construct command pipelines with two or more stages.
\item    Explain what usually happens if a program or pipeline isn't given any input to process.
\item    Explain Unix's "small pieces, loosely joined" philosophy.

\end{enumerate}
\begin{multicols}{2}
\begin{itemize}
\item cd molecules
\item wc *.pdb
\item *, ? 
\item wc -l
\item wc --help
\item wc -l *.pdb $>$ lengths
\item cat lengths
\item sort lengths
\item sort lengths $>$ sorted-lengths
\item head -1 sorted-lengths
\item sort lengths $|$ head -1
\item cd north-pacific-gyre/2012-07-03
\item wc -l *.txt 
\item wc -l *.txt $|$ sort $|$ head -5
\item ls *Z.txt
\end{itemize}
\end{multicols}
\end{frame}


\end{document}

